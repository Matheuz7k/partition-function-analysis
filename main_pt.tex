\documentclass[12pt, a4paper]{article}

\usepackage[utf8]{inputenc}
\usepackage[T1]{fontenc}
\usepackage{amsmath}
\usepackage{amssymb}
\usepackage{amsthm}
\usepackage{graphicx}
\usepackage{hyperref}
\usepackage[margin=1in]{geometry}
\setlength{\parskip}{1.5ex}
\linespread{1.2}
\newtheorem{theorem}{Teorema}[section]
\newtheorem{lemma}[theorem]{Lema}
\newtheorem{proposition}[theorem]{Proposição}
\newtheorem{corollary}[theorem-]{Corolário}
\theoremstyle{definition}
\newtheorem{definition}[theorem]{Definição}
\newtheorem{example}[theorem]{Exemplo}
\theoremstyle{remark}
\newtheorem{remark}[theorem]{Observação}

\title{\textbf{Uma Análise Abrangente da Função de Partição:\\ Das Identidades de Euler à Fórmula Assintótica de Hardy-Ramanujan}}
\author{Matheus Sales dos Santos}
\date{15 de setembro de 2025}

\begin{document}

\maketitle

\begin{abstract}
\noindent Este trabalho oferece uma exploração detalhada da função de partição de inteiros, $p(n)$, um dos objetos mais fascinantes na intersecção da combinatória e da teoria dos números. Iniciamos com as definições fundamentais e representações visuais, como os diagramas de Ferrers, para construir uma base intuitiva. Em seguida, mergulhamos no trabalho pioneiro de Leonhard Euler, que introduziu a poderosa ferramenta das funções geradoras. Investigamos o célebre Teorema dos Números Pentagonais de Euler e demonstramos como ele leva a uma elegante e eficiente fórmula de recorrência para $p(n)$. O clímax do trabalho é a apresentação da fórmula assintótica de Hardy e Ramanujan, um marco da teoria analítica dos números que descreve o crescimento de $p(n)$ com precisão notável.
\end{abstract}

\tableofcontents
\newpage

\section{Introdução: A Simplicidade Enganosa das Partições}
A matemática é repleta de problemas cuja formulação é tão simples que podem ser explicados a uma criança, mas cuja solução exige ferramentas de extraordinária sofisticação. A função de partição de inteiros, $p(n)$, é um exemplo paradigmático desse fenômeno.
\begin{definition}[Partição de um Inteiro]
Uma \textbf{partição} de um inteiro positivo $n$ é uma maneira de escrever $n$ como uma soma de inteiros positivos. A ordem das partes não importa. A função de partição, $p(n)$, conta o número de partições distintas de $n$.
\end{definition}
\begin{example}
As partições de $n=5$ são: 5, 4+1, 3+2, 3+1+1, 2+2+1, 2+1+1+1, 1+1+1+1+1. Portanto, $p(5) = 7$.
\end{example}

\section{Visualizando Partições: Diagramas de Ferrers}
Um diagrama de Ferrers representa uma partição de $n$ como um padrão de $n$ pontos organizados em linhas, onde o comprimento de cada linha corresponde a uma parte da partição. A conjugação de um diagrama (transpor linhas e colunas) prova teoremas combinatórios de forma elegante.
\begin{theorem}
O número de partições de $n$ em no máximo $k$ partes é igual ao número de partições de $n$ onde nenhuma parte é maior que $k$.
\end{theorem}

\section{A Função Geradora de Euler}
A função geradora para $p(n)$ é dada pelo produto infinito:
\begin{equation}
P(x) = \sum_{n=0}^{\infty} p(n)x^n = \prod_{k=1}^{\infty} \frac{1}{1-x^k}
\end{equation}
Esta identidade transforma um problema de contagem em um problema de análise de séries de potências.

\section{O Teorema dos Números Pentagonais}
Euler descobriu uma identidade notável para o inverso da função geradora:
\begin{theorem}[Teorema dos Números Pentagonais]
$\prod_{k=1}^{\infty} (1-x^k) = \sum_{j=-\infty}^{\infty} (-1)^j x^{j(3j-1)/2}$
\end{theorem}
Esta identidade leva a uma fórmula de recorrência eficiente para $p(n)$:
$p(n) = p(n-1) + p(n-2) - p(n-5) - p(n-7) + \cdots$

\section{A Fórmula Assintótica de Hardy e Ramanujan}
Em 1918, Hardy e Ramanujan usaram o método do círculo da análise complexa para encontrar uma fórmula assintótica para $p(n)$.
\begin{theorem}[Fórmula Assintótica]
Para $n$ grande, $p(n) \sim \frac{1}{4n\sqrt{3}} e^{\pi \sqrt{\frac{2n}{3}}}$
\end{theorem}
Esta fórmula descreve o crescimento de $p(n)$ com uma precisão impressionante.

\section{Conclusão}
O estudo da função de partição $p(n)$ ilustra a profunda conexão entre combinatória, teoria dos números e análise complexa, mostrando como diferentes áreas da matemática se unem para resolver um problema de aparência simples.

\begin{thebibliography}{9}
\bibitem{andrews} Andrews, G. E. (1976). \textit{The Theory of Partitions}. Addison-Wesley.
\bibitem{apostol} Apostol, T. M. (1976). \textit{Introduction to Analytic Number Theory}. Springer.
\bibitem{hardyram} Hardy, G. H., \& Ramanujan, S. (1918). Asymptotic Formulæ in Combinatory Analysis. \textit{Proceedings of the London Mathematical Society}, s2-17(1), 75–115.
\end{thebibliography}

\end{document}
